%%%%%%%%%%%%%%%%%%%%%%%%%%%%%%%%%%%%%%%%%%%%%%%%%%%%%%%%%%%%%%%%%%
%%%%%%%% ICML 2014 EXAMPLE LATEX SUBMISSION FILE %%%%%%%%%%%%%%%%%
%%%%%%%%%%%%%%%%%%%%%%%%%%%%%%%%%%%%%%%%%%%%%%%%%%%%%%%%%%%%%%%%%%

% Use the following line _only_ if you're still using LaTeX 2.09.
%\documentstyle[icml2014,epsf,natbib]{article}
% If you rely on Latex2e packages, like most moden people use this:
\documentclass[fleqn]{article}

% use Times
\usepackage{times}
% For figures
%\usepackage{graphicx} % more modern
\usepackage{graphicx,psfrag,pstricks,epsf,url}
%\usepackage{epsfig} % less modern
\usepackage{subfigure}
\usepackage{amsmath}
\usepackage{amsfonts}

% For citations
\usepackage{natbib}

% For algorithms
\usepackage{algorithm}
\usepackage{algorithmic}

% As of 2011, we use the hyperref package to produce hyperlinks in the
% resulting PDF.  If this breaks your system, please commend out the
% following usepackage line and replace \usepackage{icml2014} with
% \usepackage[nohyperref]{icml2014} above.
\usepackage{hyperref}

% Packages hyperref and algorithmic misbehave sometimes.  We can fix
% this with the following command.
\newcommand{\theHalgorithm}{\arabic{algorithm}}

% Employ the following version of the ``usepackage'' statement for
% submitting the draft version of the paper for review.  This will set
% the note in the first column to ``Under review.  Do not distribute.''
\usepackage{icml2014} 
% Employ this version of the ``usepackage'' statement after the paper has
% been accepted, when creating the final version.  This will set the
% note in the first column to ``Proceedings of the...''
%\usepackage[accepted]{icml2014}


% The \icmltitle you define below is probably too long as a header.
% Therefore, a short form for the running title is supplied here:
\icmltitlerunning{RNN-based Music Language Models for Improving Automatic Music Transcription}

\begin{document} 

\twocolumn[
\icmltitle{Recurrent Neural Network-based Music Language Models for Improving Automatic Music Transcription}

% It is OKAY to include author information, even for blind
% submissions: the style file will automatically remove it for you
% unless you've provided the [accepted] option to the icml2014
% package.
\icmlauthor{Your Name}{email@yourdomain.edu}
\icmladdress{Your Fantastic Institute,
            314159 Pi St., Palo Alto, CA 94306 USA}
\icmlauthor{Your CoAuthor's Name}{email@coauthordomain.edu}
\icmladdress{Their Fantastic Institute,
            27182 Exp St., Toronto, ON M6H 2T1 CANADA}

% You may provide any keywords that you 
% find helpful for describing your paper; these are used to populate 
% the "keywords" metadata in the PDF but will not be shown in the document
\icmlkeywords{Recurrent neural networks, Restricted Boltsmann Machines, Probabilistic latent component analysis, Music signal analysis, Music language models}

\vskip 0.3in
]

\begin{abstract} 
In this paper, we investigate the use of Music Language Models (MLMs) for improving Automatic Music Transcription (AMT) performance. AMT is the process of converting an acoustic music signal into a symbolic notation, and is considered to be a fundamental problem in music signal processing. The MLMs are trained on sequences of symbolic polyphonic music. We train Recurrent Neural Network (RNN)-based models, as they are capable of capturing complex temporal structure present in symbolic music data. Similar to the function of language models in automatic speech recognition, we use the MLMs to generate a prior probability for the occurrence of a sequence. The acoustic AMT model is based on probabilistic latent component analysis, and prior information from the MLM is incorporated into the transcription framework using Dirichlet priors. We test our hybrid models on a dataset of multiple-instrument polyphonic music and report a significant 3\% improvement in terms of F-measure, when compared to using an acoustic-only model.
 
\end{abstract} 

%Alternative abstract
%Automatic Music Transcription (AMT) involves automatically generating a symbolic transcription of an acoustic musical signal. The transcription can be thought of as the digitized version of the musical score corresponding to the music signal. It has been observed in previous research that a Music Language Model (MLM) which captures general structural properties of music (in the symbolic form), when used together with an AMT system, can benefit the overall quality of the transcription. In this paper, we present a novel method for making this combination using Dirichlet priors. \textit{NOTE: summary of technical details could go here}. By combining the predictions of a recently proposed RNN-RBM based polyphonic MLM with the transcriptions of a state-of-the-art PLCA based AMT system, we demonstrate improved transcription accuracy on the a dataset of multiple-instrument recordings.

\section{Introduction} 
\label{sec:introduction}

Automatic Music Transcription (AMT) involves automatically generating a symbolic representation of an acoustic musical signal \cite{Benetos2013b}. AMT is considered to be a fundamental topic in the field of music information retrieval (MIR) and has numerous applications in related fields in music technology, such as interactive music applications and computational musicology. Typically, the output of an AMT system is a \textit{piano-roll} representation, which is a two-dimensional matrix representation of a musical piece where the X-axis represents time quantized into regular intervals, and the Y-axis represents the keys of a piano in increasing pitch. A cell in this matrix is $1$ if the key represented by its X-coordinate is sounded at the time instant represented by its Y-coordinate.

The majority of recent transcription papers utilise and expand \emph{spectrogram factorisation} techniques, such as non-negative matrix factorisation (NMF) \cite{Li1999} and its probabilistic counterpart, probabilistic latent component analysis (PLCA) \cite{Smaragdis2006}. Spectrogram factorisation techniques decompose an input two-dimensional spectrogram of the audio signal into a product of spectral templates (that typically correspond to musical notes) and component activations (that indicate when each note is active at a given time frame). Spectrogram factorisation-based AMT systems include the work by Bertin et al.\ \cite{Bertin2009}, who proposed a Bayesian framework for NMF, which considers each pitch as a model of Gaussian components in harmonic positions. Benetos and Dixon \cite{Benetos2012} proposed a convolutive model based on PLCA, which supports the transcription of multiple-instrument music and supports tuning changes and frequency modulations (modelled as shifts across log-frequency). 

In terms of connectionist approaches for AMT, Nam et al. \cite{Nam2011} proposed a method where features suitable for transcribing music are learned using a deep belief network consisting of stacked restricted Boltzmann machines (RBMs). The model performed classification using support vector machines and was applied to piano music. B\"{o}ck and Schedl used recurrent neural networks (RNNs) with Long Short-Term Memory units for performing polyphonic piano transcription \cite{Bock2012}, with the system being particularly good at recognising note onsets. 

There is no doubt that a reliable acoustic model is important for generating accurate symbolic transcriptions of a given music signal. However, since music exhibits a fair amount of structural regularity much like language, it is natural for one to think of the possibility of improving transcription accuracy using a \textit{music language model} (MLM) in a manner akin to the use of a language model to improve the performance of a speech recognizer \cite{Rabiner1993}. In \cite{Boulanger-Lewandowski2012}, the predictions of a polyphonic MLM were used to this end, which was further developed in \cite{Boulanger-Lewandowski2013}, where an input/output extension of the RNN-RBM was proposed that learned to map input sequences to output sequences in the context of AMT. Another example of symbolic information which can improve the performance of acoustic models are \textit{score informed} approaches, which have been applied in music research tasks such as source separation \cite{Ewert2012}, voice separation \cite{
Ewert2011} and tonic identification \cite{Senturk2013}. % More references, if required: Source separation - Ganseman2010, Hennequin2011

In the present work, we make use of the predictions made by a Recurrent Neural Network (RNN) and a Recurrent Neural Network-Neural Autoregressive Distribution Estimator (RNN-NADE) based polyphonic MLM proposed in \cite{Boulanger-Lewandowski2012} to refine the transcriptions of a PLCA-based AMT system \cite{Benetos2012, Benetos2013}. Information from the MLM is incorporated into the PLCA-based acoustic model as a prior for pitch activations during the parameter estimation stage. It was observed that combining the two models in this way boosts transcription accuracy by +3\% on the Bach10 dataset of multiple-instrument polyphonic music \cite{Duan2010}, compared to using the acoustic AMT system only.

The outline of this paper is as follows. The PLCA-based transcription system is presented in Section \ref{sec:transcription}. The RNN-based polyphonic music prediction system that is used as a music language model is described in Section \ref{sec:prediction}. The combination of the two aforementioned systems is presented in Section \ref{sec:combination}. The employed dataset, evaluation metrics, and experimental results are shown in Section \ref{sec:evaluation}; finally, conclusions are drawn and future directions are indicated in Section \ref{sec:conclusions}.

\section{Automatic Music Transcription System} \label{sec:transcription}


For combining acoustic and music language information in an automatic transcription context, we employ the transcription model of \cite{Benetos2012}, which supports the transcription of multiple-instrument polyphonic music and also supports pitch deviations or frequency modulations. The model of \cite{Benetos2012} is based on probabilistic latent component analysis (PLCA), which is a latent variable analysis method which has been used for decomposing spectrograms \cite{Shashanka2008} and can be viewed as a probabilistic version of non-negative matrix factorization \cite{Li1999}. For computational efficiency purposes, we employ the fast implementation from \cite{Benetos2013}, which utilized pre-extracted note templates that are also pre-shifted across log-frequency, in order to account for frequency modulations or tuning changes. In addition, as was shown in \cite{Smaragdis2009}, PLCA-based models can utilise priors for estimating unknown model parameters, which will be useful in this paper for informing the 
acoustic transcription system with symbolic information.

The transcription model takes as input a normalised log-frequency spectrogram $V_{\omega,t}$ ($\omega$ is the log-frequency index and $t$ is the time index) and approximates it as a bivariate probability distribution $P(\omega,t)$. $P(\omega,t)$ is decomposed into a series of log-frequency spectral templates per pitch, instrument, and log-frequency shifting (which indicates deviation with respect to the ideal tuning), as well as probability distributions for pitch, instrument, and tuning. 

The model is formulated as:
\begin{equation}
P(\omega,t) = P(t)\sum_{p,f,s}P(\omega|s,p,f)P_{t}(f|p)P_{t}(s|p)P_{t}(p) \label{eq:Model}
\end{equation} 
where $p$ denotes pitch, $s$ denotes the musical instrument source, and $f$ denotes log-frequency shifting (which indicates tuning/pitch deviations). $P(t)$ is the energy of the log-spectrogram, which is a known quantity. $P(\omega|s,p,f)$ denote pre-extracted log-spectral templates per pitch $p$ and instrument $s$, which are also pre-shifted across log-frequency. The pre-shifting operation is made in order to account for pitch deviations, without needing to formulate a convolutive model across log-frequency. $P_{t}(f|p)$ is the time-varying log-frequency shifting distribution per pitch, $P_{t}(s|p)$ is the time-varying source contribution per pitch, and finally, $P_{t}(p)$ is the pitch activation, which essentially is the resulting music transcription. As a time-frequency representation in the log-frequency domain we use the constant-Q transform (CQT) with a log-spectral resolution of 60 bins/octave \cite{Schoerkhuber10}.

The unknown model parameters ($P_{t}(f|p)$, $P_{t}(s|p)$, $P_{t}(p)$) can be iteratively estimated using the expectation-maximisation (EM) algorithm \cite{Dempster77}. For the \emph{Expectation} step, the following posterior is computed:
 \begin{equation}
  P_{t}(p,f,s|\omega) = \frac{P(\omega|s,p,f)P_{t}(f|p)P_{t}(s|p)P_{t}(p)}{\sum_{p,f,s}P(\omega|s,p,f)P_{t}(f|p)P_{t}(s|p)P_{t}(p)} \label{eq:EStep}
 \end{equation}
  
For the \emph{Maximization} step (without using any priors) unknown model parameters are updated using the posterior computed from the Expectation step:
 \begin{equation}
 P_{t}(f|p) = \frac{\sum_{\omega,s}P_{t}(p,f,s|\omega)V_{\omega,t}}{\sum_{f,\omega,s}P_{t}(p,f,s|\omega)V_{\omega,t}} 
\end{equation}
\begin{equation}
 P_{t}(s|p) = \frac{\sum_{\omega,f}P_{t}(p,f,s|\omega)V_{\omega,t}}{\sum_{s,\omega,f}P_{t}(p,f,s|\omega)V_{\omega,t}} \label{eq:MStepInstrument}
\end{equation}
\begin{equation}
 P_{t}(p) = \frac{\sum_{\omega,f,s}P_{t}(p,f,s|\omega)V_{\omega,t}}{\sum_{p,\omega,f,s}P_{t}(p,f,s|\omega)V_{\omega,t}} \label{eq:MStepTranscription}
\end{equation}


We consider the sound state templates to be fixed, so no update rule for $P(\omega|s,p,f)$ is applied. Using fixed templates, 20-30 iterations using the update rules presented in the present section are sufficient for convergence. The output of the system is a pitch activation which is scaled by the energy of the log-spectrogram:
\begin{equation}
P(p,t) =  P(t)P_{t}(p)\label{eq:transcription}
\end{equation}
After performing 5-sample median filtering for note smoothing, thresholding is performed on $P(p,t)$ followed by minimum note duration pruning set to 40ms (corresponding to the length of one time frame) in order to convert $P(p,t)$ into a binary piano-roll representation, which is the output of the transcription system, and is also used for evaluation purposes. 

\section{Polyphonic Music Prediction System} 
\label{sec:prediction}

It was demonstrated in \cite{Boulanger-Lewandowski2012} how a music language model (MLM) can be used to improve the transcription performance of a purely acoustic model. The MLM employed there was based on the recurrent neural network-restricted Boltzmann machine (RNN-RBM). A related model --- the recurrent neural network-neural autoregressive distribution estimator (RNN-NADE) was also used for the same purpose with comparable results. In the present work, we employ both the standard RNN, and the RNN-NADE as MLMs for boosting the transcription accuracy of the PLCA based model described in the previous section. In this section, we briefly describe the RNN-NADE which we used in our work as the MLM, and the necessary background for understanding this model.

	\subsection{Recurrent Neural Network}
	\label{subsec:rnn}
	A recurrent neural network (RNN) is a powerful model for time-series data which is known to account for long-term temporal dependencies when trained effectively. Given a sequence of inputs $v_1, v_2, \ldots, v_T$ each in $\mathbb{R}^n$, the network computes a sequence of hidden states $\hat{h}_1, \hat{h}_2, \ldots, \hat{h}_T$ each in $\mathbb{R}^m$, and a sequence of predictions $\hat{y}_1, \hat{y}_2, \ldots, \hat{y}_T$ each in $\mathbb{R}^k$ by iterating the equations
	\begin{eqnarray}
		h_t & = & e(W_{\hat{h}x} v_t + W_{\hat{h}\hat{h}} \hat{h}_{t-1} + b_{\hat{h}}) \\
		\hat{y}_t & = & g(W_{y\hat{h}})
	\end{eqnarray}
	
	where $W_{y\hat{h}}$, $W_{\hat{h}x}$, $W_{\hat{h}\hat{h}}$ are the weight matrices and $b_{\hat{h}}$, $b_y$ are the biases and $e$ and $g$ are pre-defined vector valued functions which are typically non-linear and applied element-wise. The RNN also has a special initial bias $b^{init}_{\hat{h}}$ which replaces the formally undefined expression $W_{\hat{h}\hat{h}} \hat{h}_0$ at time $t = 1$. 
	
	In theory, a recurrent neural network can be easily trained using the gradient-based Back-Propagation Through Time algorithm \cite{Werbos1990} using the exactly computable error gradients in the network. However, $1^{st}$ order gradient methods fail to correctly train RNNs in certain cases. This difficulty has been associated with what is known as the \textit{vanishing/exploding gradients} phenomenon \cite{Bengio1994}, where the errors exhibit exponential decay/growth as they are back-propagated through time. Several proposals have been made to overcome this difficulty while retaining the predictive power of the RNN \cite{Hochreiter1997, Jaeger2002, Martens2011}. % Optionally, if low on space, replace preceding paragraph with the following sentence and add it to the first paragraph of this RNN subsection: Owing to the difficulty in learning long-term temporal dependencies with the RNN with the gradient-based Back-Propagation Through Time algorithm \cite{Werbos1990}, various alternatives for training RNNs have been proposed over the years \cite{Hochreiter1997, Jaeger2002, Martens2011}.
%
%	\subsection{Restricted Boltzmann Machine}
%	\label{subsec:rbm}
%	A restricted Boltzmann Machine (RBM) is an energy-based model with a bipartite structure consisting of a \textit{visible} layer $v$ (with bias parameters $b_v$), a \textit{hidden} layer $h$ (with bias $b_h$) and a weight-matrix $W_{vh}$ between these two layers \cite{Smolensky1986, Hinton2002}. The joint probability of a given configuration of the visible layer $v$ and hidden layer $h$ is given by $p(v, h) = exp(-h^T W v - b_{v}^{T} v - b_{h}^{T} h) / Z$, where $Z$ is the partition function which is usually intractable. An RBM can be trained in an unsupervised manner using the Contrastive Divergence learning algorithm \cite{Hinton2002, Tieleman2008}. Inference in RBMs involves 	


	\subsection{Neural Autoregressive Distribution Estimator}
	\label{subsec:nade}
	The neural autoregressive distribution estimator (NADE) \cite{Larochelle2011} is a graphical model inspired by the Restricted Boltzmann Machine \cite{Smolensky1986, Hinton2002}. It shares the structural properties of the RBM in that it has a visible layer $v$ (with biases $b_v$), a hidden layer $h$ (with biases $b_h$), with these two layers connected by a weight-matrix $W$. It facilitates the exact inference $p(v)$ given an input vector $v$, which is not possible in RBMs since there one has to compute the intractable \textit{partition function} \cite{Larochelle2011}. This wass made possible by thinking of the RBM as a \textit{fully visible sigmoid belief network} (FVSBN) \cite{Neal1992}. The FVSBN is a special case of a family of models known as fully visible Bayesian networks \cite{Frey1998} with the property
	\begin{equation}
		p(v) = \prod_{i=1}^D p(v_i|v_{parents(i)})
	\end{equation}
	where all observation variables $v_i$ are arranged into a directed acyclic graph and $v_{parents(i)}$ corresponds to all the variables in v that are parents of $v_i$. In an FVSBN, the acyclic graph is obtained by defining the parents of $v_i$ as all variables that are to its left, or $v_{parents(i)} = v_{<i}$ where $v_{<i}$ refers to the subvector containing all variables $v_j$ such that $j<i$. In the case of the NADE, $p(v_i|v_{parents(i)})$ can be computed as follows
	\begin{eqnarray}
 		p(v_i=1|v_{parents(i)}) & = & \sigma(b_{v}^{(i)}) + (W^T)_{i,\cdot} h_i) \\
		h_i & = & \sigma(b_h + W_{\cdot , <i} v_{<i})
	\end{eqnarray}
	
	Untying the weights $W$ and $W^T$ results in a more powerful model. In the NADE, the cost of computing $p(v)$ is $O(HD)$, where $H$ is the number of hidden units and $D$ is the dimensionality of the vector $v$.
	
	\subsection{Recurrent Neural Network-Neural Autoregressive Distribution Estimator}
	\label{subsec:rnn-nade}
	Putting together the models described in Sections \ref{subsec:rnn} and \ref{subsec:nade}, we obtain the RNN-NADE, which is a model proposed for high-dimensional time-series.


% Models for music prediction capture structure in musical data. Such models deal with music data in the symbolic form \cite{Orio2006}. Structure in music can be both sequential and parallel. For example, a melody can be thought to have purely sequential structure since . On the other hand, a polyphonic piece of music has parallel structure as well due to the added possibility of occurrence of simultaneous pitches at any given time instant.

\section{Combining Transcription and Prediction}
\label{sec:combination}

%\textit{NOTE: Could describe how the predictions made by the MLM influence the transcription of the  AMT system in this section.}

In this section, we describe the process for combining the acoustic model with the music language model for deriving an improved transcription. Firstly, the input music signal is transcribed using the process described in Section \ref{sec:transcription}. The resulting piano-roll representation of the transcription system is considered to be a sequence $v_1, v_2, \ldots, v_T$ that is placed as input to the MLM presented in Section \ref{sec:prediction}. For the RNN-NADE, we compute the probability $P(v_i|\mathbf{v_{<i}})$ for all time frames, and use that as prior information for the combined model, with the prior information  denoted as $P_{\mathit{MLM}}(p,t)$, where $P_{\mathit{MLM}}(p=i,t)=P(v_i|\mathbf{v_{<i}})$. For the RNN, the prediction output is directly denoted as $P_{\mathit{MLM}}(p,t)$, since pitch probabilities are independent.

As shown in \cite{Smaragdis2009}, PLCA-based models use multinomial distributions; since the Dirichlet distribution is conjugate to the multinomial, a Dirichlet prior can be used to enforce structure on the pitch activation distribution $P_{t}(p)$. Following the procedure of \cite{Smaragdis2009}, we define the Dirichlet hyperparameter for the pitch activation as:
\begin{equation}
 \alpha(p|t) = \frac{P(p|t)P_{\mathit{MLM}}(p,t)}{\sum_{p}P(p|t)P_{\mathit{MLM}}(p,t)}
\end{equation}
where $\alpha(p|t)$ essentially is a pitch activation probability which is filtered through a pitch indicator function computed from the symbolic prediction step (the denominator is simply for normalisation purposes).

The recording is then re-transcribed, using as aditional information the prior computed from the transcription step. The modified update for the pitch activation which replaces (\ref{eq:MStepTranscription}) is given by:
\begin{equation}
 P_{t}(p) = \frac{\sum_{\omega,f,s}P_{t}(p,f,s|\omega)V_{\omega,t}+\kappa\alpha(p|t)}{\sum_{p,\omega,f,s}P_{t}(p,f,s|\omega)V_{\omega,t}+\kappa\alpha(p|t)} \label{eq:modifiedMStepPitchActivation}
\end{equation}
where $\kappa$ is a weight parameter expressing how much the prior should be imposed; as in \cite{Smaragdis2009}, the weight decreases from 1 to 0 throughout the iterations. In a larger context, the transcription creates a symbolic prediction, which in turn improves the subsequent re-transcription of the music signal. An overview of the complete transcription-prediction system architecture can be seen in Fig. \ref{fig:system}.

\begin{figure*}
\begin{center}
\resizebox{340pt}{!}{
 \includegraphics{figures/FigSystem.eps} 
 }
 \caption{Proposed system diagram.}
 \end{center}
 \label{fig:system}
\end{figure*}


\section{Evaluation} \label{sec:evaluation}

\subsection{Dataset}

For testing the transcription system, we employ the Bach10 dataset \cite{Duan2010}, which is a freely available multi-track collection of multiple-instrument polyphonic music, suitable for multi-pitch detection experiments. It consists of ten recordings of J.S. Bach chorales, performed by violin, clarinet, saxophone, and bassoon. Pitch ground truth for each instrument is also provided. Due to the tonal and homogenous content of the dataset, it is suitable for testing the incorporation of music language models in a transcription system. For training the transcription system, pre-extracted and pre-shifted spectral templates are extracted for the instruments present in the dataset, using isolated note samples from the RWC database \cite{Goto2003}. 

\subsection{Metrics}

For evaluating the performance of the proposed system for multi-pitch detection, we employ the precision, recall, and F-measure metrics, which are commonly used in transcription evaluations \cite{MIREX}:
\begin{equation}
 \mathit{Pre} = \frac{N_{\mathit{tp}}}{N_{\mathit{sys}}},\ \
\ \mathit{Rec} = \frac{N_{\mathit{tp}}}{N_{\mathit{ref}}},\
\ \ \mathit{F} = \frac{2\cdot\mathit{Rec}\cdot\mathit{Pre}}{\mathit{Rec}+\mathit{Pre}}
\label{eq:PRF}
\end{equation}
where $N_{\mathit{tp}}$ is the number of correctly detected pitches, $N_{\mathit{sys}}$ is the number of detected pitches, and $N_{\mathit{ref}}$ is the number of ground-truth pitches. As in the public evaluations on multi-pitch detection carried out through the MIREX framework \cite{MIREX}, a detected note is considered correct is if its pitch is the same as the ground truth pitch and its onset is within a 50ms tolerance interval of the ground-truth onset.

\subsection{Results}

\section{Conclusions} \label{sec:conclusions}

In this paper, we proposed a system for automatic music transcription which incorporated prior information from a polyphonic music prediction model based on recurrent neural networks. The acoustic transcription model was based on probabilistic latent component analysis, and information from the prediction system was incorporated using Dirichlet priors. Experimental results using the Bach10 dataset of multiple-instrument recordings showed that there is a clear and significant improvement (3\% in terms of F-measure) by combining a music language model with an acoustic model for improving the performance of the latter.

In the current evaluation, the language models are trained on only one dataset. In the future, we would like to evaluate the proposed system using language models trained from different sources to see if this helps the MLMs generalize better. We will also investigate different system configurations, by bootstrapping the system for demonstrating that an improved transcription can lead to an improved prediction, and so on. We will also investigate the effect of using different RNN architectures like Long Short Term Memory (LSTM) and bi-directional RNNs and LSTMs. Finally, we would like to extend and improve the current models for high-dimensional sequences to better fit the requirements for music language modelling. 

% In the unusual situation where you want a paper to appear in the
% references without citing it in the main text, use \nocite
\nocite{langley00}

\bibliography{bibliography}
\bibliographystyle{icml2014}

\end{document} 


% This document was modified from the file originally made available by
% Pat Langley and Andrea Danyluk for ICML-2K. This version was
% created by Lise Getoor and Tobias Scheffer, it was slightly modified  
% from the 2010 version by Thorsten Joachims & Johannes Fuernkranz, 
% slightly modified from the 2009 version by Kiri Wagstaff and 
% Sam Roweis's 2008 version, which is slightly modified from 
% Prasad Tadepalli's 2007 version which is a lightly 
% changed version of the previous year's version by Andrew Moore, 
% which was in turn edited from those of Kristian Kersting and 
% Codrina Lauth. Alex Smola contributed to the algorithmic style files.  
